\chapter{Einleitung}
Energie:
\[
	E_s= \int_{0}^{T} \left|s(t)\right|^2 \di t
\]
~\\
Korrelation (Skalarprodukt):
\[
	\left\langle s_1,s_2 \right\rangle = \int_{0}^{T} s_1(t) \cdot s_2^*(t) \di t
\]
~\\
zwei Signale sind orthogonal wenn:
\[
	\left\langle s_1,s_2 \right\rangle = 0
\]
~\\
Amplitudenspektrum:
\[
	S(f) = \mathcal{F}\left\lbrace s(t) \right\rbrace = \int_{-\infty}^{\infty} s(t) \e^{-\im 2 \pi ft} \di t
\]
~\\
Parsevalsches Theorem:
\[
	\left\langle s_1,s_2 \right\rangle = \int_{-\infty}^{\infty} S_1(f) \cdot S_2^*(f) \di f
\]
~\\
Leistungsdichtespektrum $\Gamma_{ss}(f)$ (unter der Annahme, dass die Symbolfolge
mittelwertfrei, in der Leistung auf 1 normiert und unkorreliert ist):
\[
	\Gamma_{ss}(f) = \frac{1}{T_s} \left|P(f)\right|^2
\]
~\\
P-Norm:
\[
	\left||x\right||_p = \left(\sum_{i}\left|x_i\right|^p\right)^\frac{1}{p}
\]